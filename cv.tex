\documentclass{cv}
\begin{document}
\name{Domagoj Deduš}

Rudeška cesta 168

10000 Zagreb

domagoj.dedus@gmail.com

098 9501 831

Datum rođenja: 14.1.1992.
\job{PRIJAVA ZA RADNO MJESTO}{Inženjer sistemske podrške}
\section{EDUKACIJA I TRENINZI}
\datedsubsection{Elektrotehničar}{2006-2010}
1. tehnička škola Tesla
\datedsubsection{Stručni prvostupnik inženjer informacijskih tehnologija}{2010-2014}
Tehničko veleučilište u Zagrebu

Preddiplomski stručni studij informatike - smjer elektroničko poslovanje
\datedsubsection{Stručni specijalist inženjer informacijskih tehnologija}{2015-2017}
VsiTe - Visoka škola za informacijske tehnologije Zagreb

Specijalistički diplomski stručni studij informacijskih tehnologija - smjer računalni sustavi
\datedsubsection{Cisco Certified Network Associate Routing and Switching (CCNA Routing and Switching)}{2018-2019}

Cisco ID CSCO13494268
\datedsubsection{SRCE - Sveučilišni računski centar Sveučilišta u Zagrebu}{2019}
\begin{itemize}
    \setlength\itemsep{0.1cm}
    \item Programiranje u Pythonu
    \item Advanced Linux System Administration 1
\end{itemize}
\datedsubsection{Oracle Database 12c R2: Administration Workshop Ed3}{2020}

\section{RADNO ISKUSTVO}
\datedsubsection{Network operations center (NOC) operater (student) - Optima Telekom d.d.}{02/2016-04/2017}
\begin{itemize}
    \setlength\itemsep{0.1cm}
    \item rješavanje poteškoća i nadziranje pristupne, distribucijske i jezgrene telekom mreže uz pomoć Zabbix, Cacti i Grafana nadzornih sustava
    \item nadziranje ispravnosti rada podatkovnog centra - klima, poslužitelja, energetskih uređaja, protupožarnih sustava
    \item skriptiranje u Bash-u i Python-u sa svrhom automatizacije procesa i nadzora mrežnih elemenata
    \item podešavanje osnovnih postavki usmjerivača, preklopnika, DSLAM-a - poput podešavanja SNMP protokola
    \item osmislio nadzor IPTV multicast prometa korištenjem specijalne aplikacije, Linux poslužitelja i nadzornih aplikacija
    \item komuniciranje s hrvatskim i međunarodnim pružateljima internetskih usluga sa svrhom rješavanja poteškoća
    \item komuniciranje s tehničarima i ostalim IT odjelima sa svrhom rješavanja poteškoća i povećanjem efikasnosti rada NOC odjela
    \item kreiranje i rješavanje kartica kreiranih u JIRA aplikaciji
\end{itemize}
\datedsubsection{Sistem inženjer - Bulb d.o.o.}{04/2017-Trenutno}
\begin{itemize}
    \setlength\itemsep{0.1cm}
    \item administracija Linux poslužitelja uključujući sljedeće sustave: email sustave, backup sustave, sustave pohrane podataka, nadzorne sustave, RADIUS, LDAP, DHCP i DNS servise, web servise, load balancere, proxy poslužitelje
    \item korištenje virtualizacije i kontejnerizacije
    \item osmišljavanje i projektiranje visoko dostupnih sustava sa stajališta hardvera, softvera i mreže
    \item otklanjanje raznih sistemskih, aplikativnih i mrežnih poteškoća
    \item administriranje sustava baza podataka poput PostgreSQL-a, Oracle-a i MySQL-a
    \item skriptiranje u Bash-u, Python-u i Ansible-u 
    \item instaliranje, održavanje i rad s DevOps aplikacijama: Docker, Jenkins, Nexus, gitlab-ci, Sonarqube, Red Hat Openshift
    \item instaliranje, održavanje i rad s middleware servisima i poslužiteljima popust Wildfly-a i Oracle Weblogic-a
    \item interakcija sa klijentima, sudjelovanje u radionicama, pripremanje tehničke dokumentacije
    \item proaktivna edukacija u polju sistemskog inženjerstva, računalnih mreža, DevOps filozofije rada
    \item podučavanje sadašnjih i budućih kolega za radno mjesto sistem inženjera
    \item naglasak na korištenju aplikacija otvorenog koda: linux, zimbra, postfix, isc dhcp, bind dns, kvm, libvirt, grafana, graphite, prometheus, zabbix, postgresql, rocketchat, openshift okd, docker, lxc, lxd, pacemaker, corosync, apache, nginx, haproxy, openldap...
\end{itemize}
\section{VJEŠTINE}
Profinjenost u korištenju i poznavanju Linux (i Windows) operacijskih sustava i raznih aplikacija.

Profinjenost u poznavanju rada računalnih mreža i protokola.

Naglasak na automatizaciji svega što se može automatizirati, skriptiranje.

Forenzika i snažno razumijevanje rada računalnih sustava.

Timski rad.

Odličan u engleskom jeziku u govoru i pismu.

Vozačka dozvola B kategorije.

Iskustvo u vođenju tima, orijentiranost na organizaciju u radu, rezultat i kvalitetu odrađenog posla.

Kontinuirano usavršavanje stečenog znanja.

Konstantno napredovanje u polju informacijskih tehnologija i komunikacijskih vještina.
\end{document}
